% Autogenerated translation of week-1.md by Texpad
% To stop this file being overwritten during the typeset process, please move or remove this header

\documentclass[a4paper, 12pt]{book}
\usepackage{graphicx}
\usepackage[utf8]{inputenc}
\usepackage[a4paper]{geometry}
\usepackage{hyperref}
\pagestyle{plain}
\begin{document}

 $<$!--sec data-title="Week 1" data-id="section1" data-show=false ces--$>$ 

\subsection*{Monday, week 1 Gas Exchange \textbf{8:30 AM – 9:00 AM} Dave Moore: Welcome and Introduction to the Course \texttt{$<$br$>$} \textbf{9:00 AM – 9:30 AM} Flux Observations: History \& Context (Kim Novick)\texttt{$<$br$>$} \textbf{9:30 AM – 10:30 AM} Team Licor: Introduction to leaf level flux measurements (Aaron Saathoff \& Tom Avenson)\texttt{$<$br$>$} \textbf{10:30 AM – 10:45 AM} Break\texttt{$<$br$>$} \textbf{10:45 AM – 12:15 PM} Team Licor: Leaf-level flux measurements continued (Aaron Saathoff \& Tom Avenson)\texttt{$<$br$>$} **12:30 PM – 1:30 PM ** Lunch\texttt{$<$br$>$} **1:30 PM – 4:30 PM ** Hands-on Work with Infra-red gas analyzer (Measurements of A:Ci curves and A:PPFD curves on aspen leaves, Aaron Saathoff, Tom Avenson \& Dave Moore)\texttt{$<$br$>$} **4:30 PM – 5:00 PM ** Debrief gas exchange measurements (Aaron Saathoff, Kim Novick)\texttt{$<$br$>$} **5:00 – 6:00 PM ** Free Time \texttt{$<$br$>$} **6:00 PM ** Dinner **7:00 PM ** Crash test talks - Introductions to each other (3 slides / 3 minutes) [//]: \# (Hidden from students:FileTheory of Gas Exchange Measurements - Pat Morgan 2015 File) [//]: \# (Pat Morgan - Theory of Leaf-Level Gas Exchange Measurements File) \#\#\# Tuesday, week 1 - Photosynthesis to the control volume 9:00 AM – 10:15 AM Belinda Medlyn: Chloroplast- and Leaf-Level Flux Modeling (Lecture) Link to Further reading from Belinda 10:15 AM – 10:30 AM Break Some explanations from Andrew Leakey (2015) and Carl Bernacchi (2013) Carl Bernacchi's version of the Farquhar Model 10:30 AM – 12:30 PM Belinda Medyln: Modeling the Biochemistry of Photosynthesis (Hands-on computer) Link the the R package plantecophys Other packages we've used - Pecan.photosynthesis 12:30 PM – 1:30 PM Lunch 1:30 PM – 3:00 PM Russ Monson: Theory and Measurement of Canopy Fluxes Russ 3:00 PM – 3:15 PM Break 3:15 PM – 4:45 PM Ed Swiatek: Calculation of the Eddy Flux using pen and paper 4:45 PM – 6:00 PM Free Time 6:00 PM - 7:00 PM Monson’s Musings Leaf fluxes - mathematical modeling - FURTHER READING from Belinda Medlyn 2016 Page Reading material - Photosynthesis measurements (Carl) Folder Russ Monson - The Eddy Flux File Hidden from students:FileDan Yakir - Stable Isotopes and Other Tracers to Complement Flux Measurements File Hidden from students:FileDave Bowling - Stable Carbon Isotopes of Carbon Dioxide in Ecosystem Science File Hidden from students:FolderIsotopes - reading material Dan Yakir Folder Hidden from students:FolderIsotopes - Reading material Folder Farquhar Model F File PS-FIT 7.3 excel file Carl Bernacchi - Chloroplast- and Leaf-Level Flux Modeling File Hidden from students:FileRuss Monson - The Eddy Flux File Hidden from students:FolderReading Material for Russ - The Eddy Flux Folder Paper Pencil Excercise File Andrew Leakey - Measurement, analysis and interpretation of leaf photosynthetic gas exchange File Russ Monson - C4 Photosynthesis File Ed's pen and pencil exercise \& recommended papers (WPL etc) File \#\#\# Wednesday, week 1 - Eddy flux Instruments and data; uncertainty \& organization 9:00 AM – 10:30 PM Ed Swiatek (Larry Jacobsen): Eddy Flux Instrumentation – Sonics Ed Swiatek Larry 10:30 AM – 12:15 PM James Kathilankal: Eddy Flux Instrumentation – Gas Sensors 12:15 PM – 1:30 PM Lunch 1:30 PM – 3:30 PM Dario Papale - on interpreting flux data, storage estimation and gap filling (slides coming soon) (Here's a video from last year Ankur Desai spoke on some related topics including flux partitioning) Break 3:20 - More Dario Kok effect - Review by Mary Heskl \& set up groups for Thursday projects 6:00 PM – 7:00 PM Dinner This evening - Kim and I might play some music in the rec room, Ed might be convinced to tell some stories about 'interesting' flux set ups. Hidden from students:FolderPapers for Papuga Flux Network Lecture Folder NOVICK\emph{ENERGY}ET\emph{REFS File Hidden from students:FileMarcy Litvak - Fluxes Across Ecological Gradients File Desai - What to do with NEE? File NOVICK}EnergyBalanceandET File Novick\_PenmanMonteithDerviation File James Kathilankal - Gas analyzers, theory and maintenance, and site intercomparisons File Ankur Desai - You have NEE File Kim Novick - Energy balance and evapotranspiration External Media File Ed Swiatek - Eddy Covariance Instrumentation File \#\#\# Thursday, week 1 - Group Projects, Equipment Set up and free time | | | |:------------------|:------------------| | 8:30 AM – 11:30 PM| A) Campbell Scientific Instrumentation 9 Students set up Eddy Covariance \& biometeorology systems with Ed and Larry \texttt{$<$br$>$} B) Data Analysis Students Prepare Projects (groups of 4-6): Kim Novick (Energy \& Water), Dario Papale (Flux processing)| | 12:00 – 1:00 PM| Lunch | | 1:00 PM – 3:00 PM | Campbell Scientific Instrumentation Show and tell (project presentations) | | 3:00 PM – 6:00 PM | Free Time / finish presentations | | 6:00 PM – 7:00 PM | Dinner | | 7:00 PM – 9:00 PM | Data Analysis Student Presentations @ Williams Village | \#\#\# Friday, week 1 - Asking questions with flux measurements - Remote Sensing AM Marcy Litvak: Asking Ecological Questions with Flux Towers Marcy Hidden from students:FilePaul Stoy - Penman-Monteith Leaves to Canopies File Hidden from students:FolderReading Material - Water fluxes (Paul) Folder Hidden from students:FileDennis Baldocchi - Integrating Information on 'Biosphere Breathing' from Chloroplast to the Globe File Hidden from students:FileTristan Quaife - Remote Sensing for Carbon Cycle Science File Hidden from students:FileMODIStools and references File MODIStools and references Hidden from students:FileMarcy Litvak - Fluxes across ecological gradients File Hidden from students:FileTristan Quaife - Remote sensing for carbon cycle science File $<$!--endsec--$>$}

\end{document}
